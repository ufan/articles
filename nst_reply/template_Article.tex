\documentclass[]{article}

%opening
\title{Reply to NST reviewers}
\author{Zhou Yong}

\begin{document}

\maketitle

\begin{abstract}

\end{abstract}

\section{Reviewer 1}
\subsection{Question: Why not show $G_{relative}$ distribution against the reference PMT}
Original statements: P.7 - Fig.7: I believe it would be preferable to show th plot of $G_{relative}$ with respect one of the reference PMT, which remain untouched during the full set of measurements (if I understand well the procedure). 

\begin{enumerate}
	\item Reference PMT Vesus Monitoring PMT
	\item The selection procedure: 
	\item Why smallest gain: the graph is much more understandable
\end{enumerate}

\subsection{Question: More detalis about the integrating sphere and how it works as a perfect integrator for short light pulses}
Original statements: P.4 - L.36: Can you provide some more details on the integrating sphere and in particular how it works as a perfect light integrator even for relatively short duration pulse of the order of few tens of ns?

\subsection{Question: More infos about R4443}
Original statements: The paper refers to the Hamamatsu R4443 photomultiplier that however is not listed in the Hamamatsu catalogue and surfing the web I did not find a data sheet for it. If, as I believe to understand, the R4443 is a modified version of the R647 it should be clearly stated and the differences highlighted.

\begin{enumerate}
	\item Specification can't be found in Hamamatsu website
	\item The difference or specification shall be listed
\end{enumerate}


\section{Reviewer 2}

\section{Reviewer3}
Original statements:
\begin{enumerate}
	\item Check the references, some mistakes here and there
	\item Summary should be more elaborative
	\item Change the black background of Fig.3(a)
\end{enumerate}
\end{document}
